\section{Die}

\subsection{Facts}

\begin{itemize} [itemsep=2pt,parsep=2pt]
    \item Part of Game.
    \item There are two die in play in a game.
    \item Each die has 6 sides with the numbers 1 through 6.
    \item At the start of the game, both players roll to determine who goes first.
    The player with the higher roll plays first, and the dice are re-rolled if equal.
    \item If both die are equal it is called rolling doubles and both die can be played twice.
\end{itemize}

\noindent{\newline\textbf{Die} - Dice rolls determine both the order of play and the number of points a player can move their chips. The order of play is decided at the start of the game, where the player who rolls the higher number plays first. If the die are equals then the dice are re-rolled. Chips move based on the number on the die, but numbers can be combined to move further in 1 turn as long as at least 1 move is available for that die at every step. Rolling 2 of the same die allows each die to be played twice.}

\subsection{Summary}
A single Die does not contain enough information to warrant becoming a class, but it will give attributes and behaviours to \textbf{Game}.


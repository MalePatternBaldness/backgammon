\section{Point}

\subsection{Facts}

\begin{dashed}
    \item Is part of the board.
    \item Points are divided up into smaller sections of the Board, each consisting of 6 points.
    \item Triangle shaped with the base against either the top or bottom edge of the Board and alternate in colour.
    \item Can be empty or contain chips.
    \item Can only hold chips of a single colour at a time.
    \item Has no limit to the number of chips each can hold.
    \item A point occupied by no more than 1 of the opposing player’s chips is called a blot.
\end{dashed}

\noindent{\newline \textbf{Point} - A series of triangles alternating in colour that chips can be placed on. Points can only hold chips of a single colour at a time, but there is no limit to the number of chips they can hold. A point occupied by more than 1 of the opposing player's chips is locked and cannot be moved to.}

\subsection{Summary}

A point at the very least needs to be some kind of container that has the ability
to store checkers. It also needs to determine how many checkers are on a point.


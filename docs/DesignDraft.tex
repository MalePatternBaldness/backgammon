\documentclass{report}

\usepackage[utf8]{inputenc}
\usepackage[T1]{fontenc}
\usepackage[english]{babel}
\usepackage{multicol}
\usepackage{enumitem}
\usepackage[cachedir=build/minted_pymint,outputdir=build]{minted}

\newlist{dashed}{itemize}{3}
\setlist[dashed, 1]
{label=\textendash, nosep,
    leftmargin=\parindent,
    rightmargin=10pt
}

\begin{document}

\begin{titlepage}
    \begin{center}
        \vspace{1.5cm}
        
        \Large
        Computer Science 101 -- Winter 2022 \\
        Project Design Draft
        
        \vspace*{4cm}
            
        \huge
        \textbf{Backgammon Design Draft}
        \vspace*{0.4cm}

        \large
        February 1, 2022 \\
        \vspace*{0.2cm}

        \footnotesize
        v0.1.0
        \vspace{1.5cm}

        \vfill
        CPSC 101 \\
        UNBC \\
        David Casperson \\

        \vspace{0.8cm}
        Joseph Diza, Brandon Wright, Gabriel Atwood, Harrdeep Singh, Ryan Skidmore, Pridi \\
        Team Java
    \end{center}
\end{titlepage}

\tableofcontents

\chapter{Introduction}

This is an initial draft for a CPSC 101 for the Winter 2022 semester by Team Decaffeinators
with Joseph Diza as the lead author and developer.  \\

\noindent
This document gives a quick \& extremely simplified overview
of the actual API and design of the Backgammon game.

\chapter{Design Elements}

\section{List of Nouns}

\begin{multicols}{2}
    \begin{itemize}
        \item Die
        \item Doubling Cube
        \item Point
        \item Player
        \item Computer
        \item Game
    \end{itemize}

    \begin{itemize}
        \item Engine
        \item Utility
        \item Board
        \item Bar
        \item Home
        \item Chip
    \end{itemize}
\end{multicols}

\chapter{Noun Entries}

\section{Die}

\subsection{Facts}

\begin{dashed}
    \item There are two die in play in a game.
    \item At the start of the game, both players roll to determine who goes first.
The player with the higher roll plays first, and the dice are re-rolled if equal.
    \item Players move their chips according to the rolls on their dice.
        E.g Player rolls 2,5 he can choose to move one piece up 2, and another up 5,
        or one piece for a combined movement of 7 points.
    \item If the player rolls the same number on both dice, he "gets doubles", which
        allows him to play each die twice.
        E.g Player rolls 6,6 now he has 4 sixes that he can play.

\end{dashed}

\subsection{Summary}

A single Die does not contain much information to warrant becoming a class (the
only useful method for a single Die would be to generate a random integer).

\section{Doubling Cube}

\subsection{Facts}

\begin{dashed}
    \item The doubling cube controls the stakes of the game.
    \item Games begin with a stake of one, the doubling cube may be used by either player.
    \item Players at the start their turns may propose to double the stakes of the game
        with the doubling cube, the opponent can accept the stakes refuse, forcing them
        to concede the game at the current stakes.
    \item The doubling cube multiplies the stakes of the winner at the end of the game.
\end{dashed}

\subsection{Attributes}

\begin{itemize}
    \item Die Faces (2, 4, 8, 16, 32, 64)
\end{itemize}

\subsection{Behaviors}

\begin{itemize}
    \item double
\end{itemize}

\subsection{Collaborations}
    \begin{itemize}
    \item multiplier is used in Score
\end{itemize}

\section{Score}

\subsection{Facts}

\begin{dashed}
    \item The points that you score for a match are determined by the type of win you achieve.
    \item A normal win requires you bear off all your chips, or that the other player concedes. This awards one point
    \item A Gammon win requires you win before the other player manages to bear off any chips. This awards 2 points
    \item A Backgammon requires you meet the prior requirements and the enemy player still has chips on your home quadrant or the bar. This awards 3 points.
    \item The points you are awarded are multiplied by the value of the doubling cube.
\end{dashed}

\subsection{Summary}
The doubling cube is a multiplier that determines the number of points awarded
to the winner at the end of a game.

\section{Point}

\subsection{Facts}

\begin{dashed}
    \item There are 24 points total on the board.
    \item Every point can hold 6 checkers.
    \item There are 12 points on the outer board, and 12 on the home board.
    \item A checker may only move to a point if it is open (contains < 2 opponents chips
        or < 6 of the player's chips)
\end{dashed}

\subsection{Summary}

A point at the very least needs to be some kind of container that has the ability
to store checkers. It also needs to determine how many checkers are on a point.

\section{Player}

\subsection{Facts}

\begin{dashed}
    \item There are two players in a game.
    \item The player can make a series of moves
\end{dashed}

\section{Engine}

\subsection{Facts}

\begin{dashed}
    \item React to the player's selection.
    \item Assign the player the colour chip of his choice.
    \item Determine the number of sets of games to be played and begin a new game after each one ends, or until the selected set or player ends the match.
    \item Record each outcome from rolling the dice
    \item When player selects a chip suggest him of the moves available for the chip
    \item Move the chip to the desired position from the player
    \item When all chips are in the home board of the player allow him to move the chips out of the board
    \item Display points with a congratulations message to the player who wins
\end{dashed}

\subsection{Attributes}

\begin{dashed}
    \item Set of games (int)
    \item Points (int)
    \item GamePoints
    \item Dice (int)
    \item Doubling Dice (int)
    \item Outcomes (Arraylist)
    \item chips/point (Array of Arraylist[24])
    \item Home
\end{dashed}

\subsection{Behaviours}

\begin{dashed}
    \item Store number of games
    \item Store Incremented points
    \item Store highest dice number when selected doubling
    \item Store random number created
    \item Store doubling dice number created
    \item Keep Record of the outcomes of dice
    \item Record chips per point
    \item Store number of chips cleared
\end{dashed}

\subsection{Collaborations}

    \item Collaborate with die and doubling cube to display random number generated between 1 and 6
    \item Collaborate with Points to keep track of number of chips per point to give the player suggestions errors if any and a notification when he wins the game
    \item Collaborate with Player and Computer to keep track of the options they select

\section{Computer}

\subsection{Facts}

\begin{dashed}
    \item Select the dice option whenever the turn comes
    \item Choose which chip to move so that AI can win the game
\end{dashed}

\subsection{Attributes}

    \item none

\subsection{Behaviours}

    \item none

\subsection{Collaborations}

    \item none

\end{document}

\section{Chip}

\subsection{Facts}
\begin{dashed}
	\item Moves clockwise based on dice (from player home quadrant).
	\item Allowed to choose which chip to move.
	\item Not allowed to move to a position where there are already 2 enemy chips.
	\item Allowed to land on a position where there is only one enemy chip, doing so
      removes that enemy chip and relocates it to the bar.
	\item Removed chips must be played next, they are played from the most
      counterclockwise position (the enemy home quadrant).
	\item If no moves are available, player pass.
	\item Each player has 15 chips
\end{dashed}

\subsection{Attributes}
\begin{dashed}
	\item Chips come in 2 colors (red \& white)
\end{dashed}

\subsection{Summary}
Due to the need to keep track of the chip location, a method maybe needed to update how many chips
are on a pointer.

\end{document}

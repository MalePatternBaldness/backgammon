\documentclass{report}

\usepackage[utf8]{inputenc}
\usepackage[T1]{fontenc}
\usepackage[english]{babel}
\usepackage{multicol}
\usepackage{enumitem}
\usepackage[cachedir=build/minted_pymint,outputdir=build]{minted}

\newlist{dashed}{itemize}{3}
\setlist[dashed, 1]
%label=\textendash, nosep
{label=\textendash, nosep,
    leftmargin=\parindent,
    rightmargin=10pt
}

\begin{document}

\begin{titlepage}
    \begin{center}
        \vspace{1.5cm}
        
        \Large
        Computer Science 101 -- Winter 2022 \\
        Project Design Draft
        
        \vspace*{4cm}
            
        \huge
        \textbf{Backgammon Design Draft}
        \vspace*{0.4cm}

        \large
        February 1, 2022 \\
        \vspace*{0.2cm}

        \footnotesize
        v0.1.0
        \vspace{1.5cm}

        \vfill
        CPSC 101 \\
        UNBC \\
        David Casperson \\

        \vspace{0.8cm}
        Joseph Diza, Brandon Wright, Gabriel Atwood, Harrdeep Singh, Ryan Skidmore, Pridi \\
        Team Java
    \end{center}
\end{titlepage}

\tableofcontents

\chapter{Introduction}

This is an initial draft for a CPSC 101 for the Winter 2022 semester by Team Decaffeinators
with Joseph Diza as the lead author and developer.  \\

\noindent
This document gives a quick \& extremely simplified overview
of the actual API and design of the Backgammon game.

\chapter{Design Elements}

\section{List of Nouns}

\begin{multicols}{2}
    \begin{itemize}
        \item Die
        \item Doubling Cube
        \item Point
        \item Player
        \item Computer
        \item Game
    \end{itemize}

    \begin{itemize}
        \item Engine
        \item Utility
        \item Board
        \item Bar
        \item Home
        \item Chip
    \end{itemize}
\end{multicols}

\chapter{Noun Entries}

\section{Die}

\subsection{Facts}

%\begin{itemize}[label=\textendash, nosep]
\begin{dashed}
    \item There are two die in play in a game.
    \item At the start of the game, both players roll to determine who goes first.
The player with the higher roll plays first, and the dice are re-rolled if equal.
    \item Players move their chips according to the rolls on their dice.
        E.g Player rolls 2,5 he can choose to move one piece up 2, and another up 5,
        or one piece for a combined movement of 7 points.
    \item If the player rolls the same number on both dice, he "gets doubles", which
        allows him to play each die twice.
        E.g Player rolls 6,6 now he has 4 sixes that he can play.

\end{dashed}
%\end{itemize}

\subsection{Summary}

A single Die does not contain much information to warrant becoming a class (the
only useful method for a single Die would be to generate a random integer).

\section{Doubling Cube}

\subsection{Facts}
\begin{dashed}
\item The doubling cube governs the stakes of the game
\item the game starts with 1 stake and the doubling cube is in the possession of both players
\item at the start of your turn you may propose to double the stakes of the game with the doubling cube, the opposing player can accept the cube and continue with double the stakes or refuse, forcing them to concede.
\item when accepted, the doubling cube is moved into the accepting players possesion, and will alternate until the end of the game.
\item at the end of the game the value of your win is multiplied by the stakes.
\end{dashed}

\subsection{Attributes}
\begin{itemize}
\item multiplier. 2, 4, 8, 16....
\item ownership
\end{itemize}

\subsection{Behaviors}
\begin{itemize}
\item double
\end{itemize}

\subsection{Collaborations}
\begin{itemize}
\item multiplier is used in Score
\end{itemize}

\section{Score}
\begin{dashed}
\item The points that you score for a match are determined by the type of win you achieve.
\item A normal win requires you bear off all your chips, or that the other player concedes. This awards one point
\item A Gammon win requires you win before the other player manages to bear off any chips. This awards 2 points
\item A Backgammon requires you meet the prior requirements and the enemy player still has chips on your home quadrant or the bar. This awards 3 points.
\item The points you are awarded are multiplied by the value of the doubling cube.
\end{dashed}

\subsection{Summary}
Could maybe be handled somewhere else like in a victory class. do we got one of those?

\section{Point}

%\begin{itemize}[label=\textendash, nosep]
\begin{dashed}
    \item There are 24 points total on the board.
    \item Every point can hold 6 checkers.
    \item There are 12 points on the outer board, and 12 on the home board.
    \item A checker may only move to a point if it is open (contains < 2 opponents chips
        or < 6 of the player's chips)
\end{dashed}
%\end{itemize}

\subsection{Summary}

A point at the very least needs to be some kind of container that has the ability
to store checkers. It also needs to determine how many checkers are on a point.

\section{Player}

\begin{dashed}
    \item There are two players in a game.
    \item The player can make a series of moves
\end{dashed}

\section{Bar}

\begin{dashed}
  \item Separates players' home boards from their outer boards.
  \item A chip goes on the bar if another players chip lands on it.
  \item Players must remove all chips off bar before they can continue playing.
  \item If chip has no where to land off of the bar, it must stay on the bar until dice roll permits them to move.
\end{dashed}

\subsection{Summary}

   \Must be able to work with chip.

\end{document}
